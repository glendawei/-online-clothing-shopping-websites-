\documentclass[12pt,a4paper]{article}

\usepackage{amsmath}
\usepackage{amsthm}
\usepackage{amssymb}
\usepackage{booktabs}
\usepackage{graphicx}
\usepackage{color}
\usepackage{fullpage}
\usepackage{listings}
\usepackage{color}
\usepackage{url}
\usepackage{multirow}
% 為了讓表格緊密
\usepackage{float}
% 為了首行空格
%\usepackage{indentfirst}
% % 為表格添加 footnote
% \usepackage{tablefootnote}

% for Chinese
\usepackage{fontspec} % 加這個就可以設定字體
\usepackage[BoldFont, SlantFont]{xeCJK} % 讓中英文字體分開設置
\setCJKmainfont{cwTeXMing} % 設定中文為系統上的字型,而英文不去更動,使用原TeX\字型
\renewcommand{\baselinestretch}{1.3}
\usepackage{xeCJK} %讓中英文字體分開設置

\parskip=5pt
\parindent=24pt
\newtheorem{lemma}{Lemma}
\newtheorem{ques}{Question}
\newtheorem{prop}{Proposition}
\newtheorem{defn}{Definition}
\newtheorem{rmk}{Remark}
\newtheorem{note}{Note}
\newtheorem{eg}{Example}
\newtheorem{aspt}{Assumption}

\definecolor{emphOrange}{RGB}{247, 80, 0}
\definecolor{stringGray}{RGB}{109, 109, 109}
\definecolor{commentGreen}{RGB}{0, 96, 0}
\definecolor{mygreen}{rgb}{0,0.6,0}
\definecolor{mygray}{rgb}{0.5,0.5,0.5}
\definecolor{mymauve}{rgb}{0.58,0,0.82}

\lstset{
  belowcaptionskip=1\baselineskip,
  breaklines=true,
  frame=L,
%  xleftmargin=\parindent,
  language = SQL,
  showstringspaces=false,
  basicstyle = \ttfamily, 
  keywordstyle = \bfseries\color{blue}, 
  emph = {symbol1, symbol2},
  emphstyle = \color{red},
  emph = {[2]symbol3, symbol4},
  emphstyle = {[2]\color{emphOrange}},
  commentstyle = \color{commentGreen}, 
  stringstyle = \color{stringGray}, 
%  backgroundcolor = \color{white}, 
%  numbers = left, % 沒有行號,複製貼上測試程式會比較方便
%  numberstyle = \normalsize, 
%	stepnumber = 1, 
%  numbersep = 10pt, 
%  title = ,
}

% "摘要", "表", "圖", "參考文獻"
\renewcommand{\abstractname}{\bf 摘要}
\renewcommand{\tablename}{表}
\renewcommand{\figurename}{圖}
\renewcommand{\refname}{\bf 參考文獻}

\begin{document}

\begin{center}
\textbf{\Large Web App Final Project \\[5pt]
Data Dictionary} \\[10pt]
2024/12/04
\end{center}

% IMAGE
\begin{table}[H]
    \centering
    \resizebox{1\columnwidth}{!}{
    \begin{tabular}{llllll}
    \toprule
        Column Name & Meaning & Data Type & Key & Constraint & Domain \\ 
    \midrule
        filename & 照片檔名 & varchar(20) & PK & Not Null & ~ \\ 
        path & 照片相對static資料夾之存放位址 & varchar(50) & ~ & Not Null & ~ \\ 
    \bottomrule
    \end{tabular}}
    \caption{資料表 IMAGE 的欄位資訊} 
\end{table}

% CLOTHES
\begin{table}[H]
    \centering
    \resizebox{1\columnwidth}{!}{
    \begin{tabular}{llllll}
    \toprule
        Column Name & Meaning & Data Type & Key & Constraint & Domain \\ 
    \midrule
        clothes\_id & 衣物編號 & bigint & PK & Not Null & ~ \\ 
        name & 衣物名稱 & varchar(20) & ~ & Not Null & ~ \\ 
        part & 部件(褲子、上衣等) & char & ~ & Not Null & \{'T', 'B', 'O', 'D', 'A'\} \\ 
        gender & 適穿性別 & char & ~ & Not Null & \{'M', 'F', 'B'\} \\ 
        price & 售價 & int & ~ & Not Null & ~ \\ 
        description & 補充敘述 & varchar(50) & ~ & ~ & ~ \\ 
    \bottomrule
    \end{tabular}}
    \caption{資料表 CLOTHES 的欄位資訊} 
\end{table}

% CLOTHES_COLOR
\begin{table}[H]
    \centering
    \resizebox{1\columnwidth}{!}{
    \begin{tabular}{llllll}
    \toprule
        Column Name & Meaning & Data Type & Key & Constraint & Domain \\ 
    \midrule
        clothes\_id & 衣物編號 & bigint & PK, FK: CLOTHES(clothes\_id) & Not Null & ~ \\ 
        color & 顏色 & varchar(2) & PK & Not Null & ~ \\ 
        image\_filename & 商品照檔名 & varchar(20) & FK: IMAGE(filename), Default("default\_img.png") & Not Null & ~ \\ 
    \midrule
        Referential triggers & ~ & On Delete & On Update & ~ \\ 
    \midrule
        \multicolumn{2}{l}{clothes\_id: CLOTHES(clothes\_id)} & Cascade & Cascade \\ 
        \multicolumn{2}{l}{image\_filename: IMAGE(filename)} & Set Default & Cascade \\ 
    \bottomrule
    \end{tabular}}
    \caption{資料表 CLOTHES\_COLOR 的欄位資訊} 
\end{table}

% CLOTHES_COLOR_SIZE
\begin{table}[H]
    \centering
    \resizebox{1\columnwidth}{!}{
    \begin{tabular}{llllll}
    \toprule
        Column Name & Meaning & Data Type & Key & Constraint & Domain \\ 
    \midrule
        clothes\_id & 衣物編號 & bigint & PK, FK: CLOTHES\_COLOR(clothes\_id) & Not Null & ~ \\ 
        color & 顏色 & varchar(2) & PK, FK: CLOTHES\_COLOR(color) & Not Null & ~ \\ 
        size & 尺寸 & varchar(2) & PK & Not Null & \{"XS", "S", "M", "L", "XL", "NA"\} \\ 
        stock\_qty & 存貨數量 & int & ~ & Not Null & ~ \\ 
    \midrule
        Referential triggers & ~ & On Delete & On Update & ~ \\ 
    \midrule
        \multicolumn{2}{l}{clothes\_id: CLOTHES\_COLOR(clothes\_id)} & Cascade & Cascade \\ 
        \multicolumn{2}{l}{color: CLOTHES\_COLOR(color)} & Cascade & Cascade \\ 
    \bottomrule
    \end{tabular}}
    \caption{資料表 CLOTHES\_COLOR\_SIZE 的欄位資訊} 
\end{table}

% USER
\begin{table}[H]
    \centering
    \resizebox{1\columnwidth}{!}{
    \begin{tabular}{llllll}
    \toprule
        Column Name & Meaning & Data Type & Key & Constraint & Domain \\ 
    \midrule
        user\_id & 使用者編號 & bigint & PK & Not Null & ~ \\ 
        fname & 名字 & varchar(10) & ~ & Not Null & ~ \\ 
        lname & 姓氏 & varchar(10) & ~ & Not Null & ~ \\ 
        password & 密碼 & varchar(20) & ~ & Not Null & ~ \\ 
        phone & 連絡電話 & varchar(20) & ~ & ~ & ~ \\ 
        email & 連絡信箱 & varchar(40) & ~ & ~ & ~ \\ 
        bdate & 生日 & date & ~ & ~ & ~ \\ 
        gender & 性別 & char & ~ & Not Null & \{'M', 'F', 'O'\} \\ 
        role & 使用者身分 & char & ~ & Not Null & \{'U', 'A'\} \\ 
    \bottomrule
    \end{tabular}}
    \caption{資料表 USER 的欄位資訊} 
\end{table}

% USER_IMAGE
\begin{table}[H]
    \centering
    \resizebox{1\columnwidth}{!}{
    \begin{tabular}{llllll}
    \toprule
        Column Name & Meaning & Data Type & Key & Constraint & Domain \\ 
    \midrule
        user\_id & 使用者編號 & bigint & PK, FK: USER(user\_id) & Not Null, Default(0) & ~ \\ 
        image\_filename & 個人照檔名 & varchar(20) & PK, FK: IMAGE(filename) & Not Null, Default("default\_img.png") & ~ \\ 
        upload\_date & 上傳日期 & timestamp & ~ & Not Null & ~ \\ 
    \midrule
        Referential triggers & ~ & On Delete & On Update & ~ \\ 
    \midrule
        \multicolumn{2}{l}{user\_id: USER(user\_id)} & Set Default & Cascade \\ 
        \multicolumn{2}{l}{image\_filename: IMAGE(filename)} & Set Default & Cascade \\ 
    \bottomrule
    \end{tabular}}
    \caption{資料表 USER\_IMAGE 的欄位資訊} 
\end{table}

% ORDER
\begin{table}[H]
    \centering
    \resizebox{1\columnwidth}{!}{
    \begin{tabular}{llllll}
    \toprule
        Column Name & Meaning & Data Type & Key & Constraint & Domain \\ 
    \midrule
        order\_id & 訂單編號 & bigint & PK & Not Null & ~ \\ 
        user\_id & 使用者編號 & bigint & FK: USER(user\_id) & Not Null, Default(0) & ~ \\ 
        sub\_total & 商品總價小計 & int & ~ & Not Null & ~ \\ 
        shipping\_fee & 運費 & int & ~ & Not Null & ~ \\ 
        payment\_type & 付款方式 & varchar(2) & ~ & Not Null & \{"V", "M", "P"\} \\ 
        address & 配送地址 & varchar(50) & ~ & Not Null & ~ \\ 
        order\_date & 下訂日期 & date & ~ & Not Null & ~ \\
        ideal\_rcv\_date & 預計送達日期 & date & ~ & ~ & ~ \\
    \midrule
        Referential triggers & ~ & On Delete & On Update & ~ \\ 
    \midrule
        \multicolumn{2}{l}{user\_id: USER(user\_id)} & Set Default & Cascade \\ 
    \bottomrule
    \end{tabular}}
    \caption{資料表 ORDER 的欄位資訊} 
\end{table}

% ORDER_STATUS_RECORD
\begin{table}[H]
    \centering
    \resizebox{1\columnwidth}{!}{
    \begin{tabular}{llllll}
    \toprule
        Column Name & Meaning & Data Type & Key & Constraint & Domain \\ 
    \midrule
        order\_id & 訂單編號 & bigint & PK, FK: ORDER(order\_id) & Not Null, Default(0) & ~ \\ 
        status & 訂單狀態 & char & PK & Not Null & \{'P', 'C', 'R'\} \\ 
        status\_date & 對應訂單狀態之日期 & timestamp & PK & Not Null & ~ \\ 
        status\_description & 補充描述 & varchar(50) & PK & Not Null & ~ \\ 
    \midrule
        Referential triggers & ~ & On Delete & On Update & ~ \\ 
    \midrule
        \multicolumn{2}{l}{order\_id: ORDER(order\_id)} & Set Default & Cascade \\ 
    \bottomrule
    \end{tabular}}
    \caption{資料表 ORDER\_STATUS\_RECORD 的欄位資訊} 
\end{table}

% ORDER_CONTAINS
\begin{table}[H]
    \centering
    \resizebox{1\columnwidth}{!}{
    \begin{tabular}{llllll}
    \toprule
        Column Name & Meaning & Data Type & Key & Constraint & Domain \\ 
    \midrule
        order\_id & 訂單編號 & bigint & PK, FK: ORDER(order\_id) & Not Null, Default(0) & ~ \\ 
        clothes\_id & 衣物編號 & bigint & PK, FK: CLOTHES\_COLOR\_SIZE(clothes\_id) & Not Null, Default(0) & ~ \\ 
        color & 顏色 & varchar(2) & PK, FK: CLOTHES\_COLOR\_SIZE(color) & Not Null, Default("NA") & ~ \\ 
        size & 尺寸 & varchar(2) & PK, FK: CLOTHES\_COLOR\_SIZE(size) & Not Null, Default("NA") & \{"XS", "S", "M", "L", "XL", "NA"\} \\ 
        purchase\_qty & 購買數量 & int & Not Null & ~ \\
    \midrule
        Referential triggers & ~ & On Delete & On Update & ~ \\ 
    \midrule
        \multicolumn{2}{l}{order\_id: ORDER(order\_id)} & Set Default & Cascade \\ 
        \multicolumn{2}{l}{clothes\_id: CLOTHES\_COLOR\_SIZE(clothes\_id)} & Set Default & Cascade \\ 
        \multicolumn{2}{l}{color: CLOTHES\_COLOR\_SIZE(color)} & Set Default & Cascade \\ 
        \multicolumn{2}{l}{size: CLOTHES\_COLOR\_SIZE(size)} & Set Default & Cascade \\ 
    \bottomrule
    \end{tabular}}
    \caption{資料表 ORDER\_CONTAINS 的欄位資訊} 
\end{table}

% BAG
\begin{table}[H]
    \centering
    \resizebox{1\columnwidth}{!}{
    \begin{tabular}{llllll}
    \toprule
        Column Name & Meaning & Data Type & Key & Constraint & Domain \\ 
    \midrule
        user\_id & 使用者編號 & bigint & PK, FK: USER(user\_id) & Not Null, Default(0) & ~ \\ 
        clothes\_id & 衣物編號 & bigint & PK, FK: CLOTHES\_COLOR\_SIZE(clothes\_id) & Not Null, Default(0) & ~ \\ 
        color & 顏色 & varchar(2) & PK, FK: CLOTHES\_COLOR\_SIZE(color) & Not Null, Default("NA") & ~ \\ 
        size & 尺寸 & varchar(2) & PK, FK: CLOTHES\_COLOR\_SIZE(size) & Not Null, Default("NA") & \{"XS", "S", "M", "L", "XL", "NA"\} \\ 
        purchase\_qty & 加入購物袋數量 & int & Not Null & ~ \\
    \midrule
        Referential triggers & ~ & On Delete & On Update & ~ \\ 
    \midrule
        \multicolumn{2}{l}{user\_id: USER(user\_id)} & Set Default & Cascade \\ 
        \multicolumn{2}{l}{clothes\_id: CLOTHES\_COLOR\_SIZE(clothes\_id)} & Set Default & Cascade \\ 
        \multicolumn{2}{l}{color: CLOTHES\_COLOR\_SIZE(color)} & Set Default & Cascade \\ 
        \multicolumn{2}{l}{size: CLOTHES\_COLOR\_SIZE(size)} & Set Default & Cascade \\ 
    \bottomrule
    \end{tabular}}
    \caption{資料表 BAG 的欄位資訊} 
\end{table}

% FAVORITE
\begin{table}[H]
    \centering
    \resizebox{1\columnwidth}{!}{
    \begin{tabular}{llllll}
    \toprule
        Column Name & Meaning & Data Type & Key & Constraint & Domain \\ 
    \midrule
        user\_id & 使用者編號 & bigint & PK, FK: USER(user\_id) & Not Null, Default(0) & ~ \\ 
        clothes\_id & 衣物編號 & bigint & PK, FK: CLOTHES\_COLOR(clothes\_id) & Not Null, Default(0) & ~ \\ 
        color & 顏色 & varchar(2) & PK, FK: CLOTHES\_COLOR(color) & Not Null, Default("NA") & ~ \\ 
    \midrule
        Referential triggers & ~ & On Delete & On Update & ~ \\ 
    \midrule
        \multicolumn{2}{l}{user\_id: USER(user\_id)} & Set Default & Cascade \\ 
        \multicolumn{2}{l}{clothes\_id: CLOTHES\_COLOR(clothes\_id)} & Set Default & Cascade \\ 
        \multicolumn{2}{l}{color: CLOTHES\_COLOR(color)} & Set Default & Cascade \\ 
    \bottomrule
    \end{tabular}}
    \caption{資料表 FAVORITE 的欄位資訊} 
\end{table}

% TRY_ON
\begin{table}[H]
    \centering
    \resizebox{1\columnwidth}{!}{
    \begin{tabular}{llllll}
    \toprule
        Column Name & Meaning & Data Type & Key & Constraint & Domain \\ 
    \midrule
        user\_id & 使用者編號 & bigint & PK, FK: USER(user\_id) & Not Null, Default(0) & ~ \\ 
        clothes\_id & 衣物編號 & bigint & PK, FK: CLOTHES\_COLOR(clothes\_id) & Not Null, Default(0) & ~ \\ 
        color & 顏色 & varchar(2) & PK, FK: CLOTHES\_COLOR(color) & Not Null, Default("NA") & ~ \\ 
        image\_filename & 試穿照片檔名 & varchar(20) & PK, FK: IMAGE(filename) & Not Null, Default("default\_img.png") & ~ \\ 
        try\_on\_date & 試穿日期 & timestamp & Not Null & ~ \\
    \midrule
        Referential triggers & ~ & On Delete & On Update & ~ \\ 
    \midrule
        \multicolumn{2}{l}{user\_id: USER(user\_id)} & Set Default & Cascade \\ 
        \multicolumn{2}{l}{clothes\_id: CLOTHES\_COLOR(clothes\_id)} & Set Default & Cascade \\ 
        \multicolumn{2}{l}{color: CLOTHES\_COLOR(color)} & Set Default & Cascade \\ 
        \multicolumn{2}{l}{image\_filename: IMAGE(filename)} & Set Default & Cascade \\ 
    \bottomrule
    \end{tabular}}
    \caption{資料表 TRY\_ON 的欄位資訊} 
\end{table}

\end{document}

\end{document}
